\documentclass[a4paper, 11pt, fleqn]{article}
\usepackage{comment} % enables the use of multi-line comments (\ifx \fi) 
\usepackage{lipsum} %This package just generates Lorem Ipsum filler text. 
\usepackage{fullpage} % changes the margin
\usepackage{amssymb}
\usepackage{amsmath}
\usepackage{graphicx}
\usepackage[square, numbers, comma, sort&compress]{natbib}  % Use the "Natbib" style for the references in the Bibliography

\begin{document}
%Header-Make sure you update this information!!!!
\noindent
\large\textbf{Deep Learning} \hfill \textbf{Ehsan Mahmoudi} \\
\normalsize Assignment 1 \hfill   \\
Dr Bokaei \hfill  \\

\section{Activation Functions}

\subsection{tanh}

Here is the formula for $tanh$: 
\begin{equation}
    \frac{e^x - e^{-x}}{e^x + e^{-x}}
\end{equation}

\begin{figure}[!htb]
    \centering
    \includegraphics{images/TanhReal.jpg}
    \caption{$tanh$ Plot}
\end{figure}

And here is the derivative formula: 
\begin{equation}
    \frac{d tanh(x)}{dx} = sech^2(x) = \left( \frac{2}{e^x + e^{-x}} \right)^2
\end{equation}

\begin{figure}[!htb]
    \centering
    \includegraphics{images/tanhderivative.JPG}
    \caption{$tanh$ Derivative Plot}
\end{figure}

\subsection{Sigmoid}
Here is the formula for $\sigma$: 
\begin{equation}
    \sigma(x) = \frac{1}{1 + e^{-x}}
\end{equation}

\begin{figure}[!htb]
    \centering
    \includegraphics[scale=0.3]{images/640px-Logistic-curve.png}
    \caption{$\sigma(x)$ Plot}
\end{figure}

And here is the derivative formula: 
\begin{equation}
    \frac{d \sigma(x)}{dx} =  \frac{e^{-x}}{(1 + e^{-x})^2}
\end{equation}

\begin{figure}[!htb]
    \centering
    \includegraphics[scale=0.8]{images/sigmoidderivative.JPG}
    \caption{$\sigma(x)$ Derivative Plot}
\end{figure}

If we want to compare the above two functions $\sigma$ and $tanh$ it seems $tanh$ is preferred because it is symmetric around 0. Having data of a layer (even hidden layers) to be around zero helps more with training. because of this  also it is advised to have input data also normalized. Also having  zero mean inputs to a layer makes more sense on regularizations that we apply on the weights, otherwise the layers with non-zero mean input will suffer more from regularizations. See \cite{LeCun:1998:EB:645754.668382}
\section{ReLU}
Here is the formula for ReLU activation function. 

\begin{equation}
    f(x) = x^{+} = max(0, x)
\end{equation}

\begin{figure}[!htb]
    \centering
    \includegraphics[scale=0.3]{images/ReLU.jpg}
    \caption{ReLU Plot}
\end{figure}

The problem with ReLU is that is not derivativable. And we need to calculate the derivative for backpropagation training. So we need to come up with a close function to be derivativable. There are two options available either calculate the gradient by numerical method or estimate the function using an analytic function such a soft-ReLU. 
On the other hand there are multiple advantages associated with ReLU.  First of all it is more aligned with what we know from behaviour of biological neural systems. Also using ReLU activation function will result in  more sparse representations of the data. This is useful specially when the network is deep. In practice also it is performing better on pre-training models such Boltzmann Machines or deep belief networks. 

Also in ReLU as the output does not saturate, we have use some sort of regularization scheme to control the weight sizes. As one of the advantages of the ReLU unit is sparsity, it makes sense to use $L_1$ regularization which also promotes the sparsity. See \cite{pmlr-v15-glorot11a}



\bibliographystyle{unsrtnat}  % Use the "unsrtnat" BibTeX style for formatting the Bibliography
\bibliography{Bibliography}  % The references (bibliography) information are stored in the file named "Bibliography.bib"

\end{document}
